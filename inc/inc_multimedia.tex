\section{Multimeida}
\softwareframe{Blender}{Professionelle 3D-Animation}{
\begin{itemize}
\item Steile Lernkurve
\item Mächtig
\item Vektoranimation
\item Videoschnitt
\item Modular
\item Open Movies: \texttt{Caminandes, Cosmos Laundromat, Tears of Steel, Big Buck Bunny, Sintel, Glass Half, Elephants Dream, Spiel: Yo Frankie }
\end{itemize}
}{blender}{blender}{}{}{}
{https://www.blender.org/}

\softwareframe{Gimp}{Viele Funktionen, gut versteckt}{
\begin{itemize}
\item Grauenhafte GUI
\item De facto Standard
\item Furchtbare GUI
\item Viele Funktionen
\item Die GUI ist ein Krampf
\item Unterstützt viele Formate
\item Mittlerweile Einfenstermodus
\item Alternativen: \texttt{Paint.NET, Krita, Wine + AffinityFoto/Photoshop :-(}
\end{itemize}
}{gimp}{gimp}{}{}{}
{https://www.gimp.org/}

\softwareframe{Inkscape}{Vektor-Grafikprogramm}{
\begin{itemize}
\item De facto Standard
\item Vektorisierung
\item Mittlere Qualität
\item Kommerzielle Alternative von Studi Ghibli: \texttt{OpenToonz}
\end{itemize}
}{inkscape}{inkscape}{}{}{}
{https://inkscape.org/}

\softwareframe{OpenSCAD}{3D für Coder}{
\begin{itemize}
\item 3D per Skript
\item Parametrisierung
\item Maker-Szene: 3D Drucker
\item Export-Formate: STL, OFF, AMF, DXF, SVG, CSG
\end{itemize}
}{openscad}{openscad}{}{}{}
{http://www.openscad.org/}

\softwareframe{Audacity}{Soundrecorder}{
\begin{itemize}
\item Viele Funktionen
\item Sound-Analyse
\item Aufnahmen
\item Retiming
\item Effekte
\end{itemize}
}{audacity}{audacity}{}{}{}
{http://www.audacityteam.org/}

\softwareframe{Ardour}{Mächtige DAW}{
\begin{itemize}
\item Komplexes Editing
\item Mixing
\item Video Sync
\item VST und VT2 Plugin Support
\end{itemize}
}{ardour}{ardour}{}{}{}
{https://ardour.org/}

\softwareframe{Kodi}{Mediacenter}{
\begin{itemize}
\item Umfangreiches Mediencenter
\item Extrem viele Plugins
\item Sehr aktiv entwickelt
\item Eigene Distros: Kodibuntu, OpenElec (RPi)
\item Früher XBMC (XBox 1)
\end{itemize}
}{kodi}{kodi}{}{}{}
{https://kodi.tv/}

\softwareframe{Darktable}{RAW Photo Workflow unter Linux}{
\begin{itemize}
\item Ähnlich viele Funktionen wie Lightroom
\item GPU Support
\item Leicht andere Bedienkonzepte
\item Alternativen: \texttt{digiKam}
\end{itemize}
}{darktable}{darktable}{}{}{}
{http://www.darktable.org/}

\softwareframe{HandBrake}{DVDs, BRs etc. rippen}{
\begin{itemize}
\item Umständliche GUI
\item Viele Automatisierungs-Funktionen
\end{itemize}
}{handbrake}{handbrake}{}{}{}
{https://handbrake.fr/}

\softwareframe{Audex}{AUdio-CDs rippen}{
\begin{itemize}
\item Cover
\item CDDB
\item Encoding-Profile
\item Empfohlene Pakete: \texttt{lame flac faac vorbis-tools mppenc} 
\end{itemize}
}{audex}{audex}{}{}{}
{https://handbrake.fr/}

\softwareframe{Kdenlive}{Videobearbeitung}{
\begin{itemize}
\item Umfangreiche Funktionen
\item Multi-Tracks
\item Proxy-Clips
\item Aktiv entwickelt
\item Manchmal instabil
\item Alternativen: \texttt{Pitivi, OpenShot, Lightworks*}
\item Ergänzend: \texttt{Natron, Davinci Resolv*}
\end{itemize}
}{kdenlive}{kdenlive}{}{}{}
{https://kdenlive.org}

\softwareframe{Gwenview}{Bildbetrachter}{
\begin{itemize}
\item KDE Projekt
\item Video Support
\end{itemize}
}{gwenview}{gwenview}{}{}{}
{https://userbase.kde.org/Gwenview}

\softwareframe{Banshee}{Einfacher Medienspieler}{
\begin{itemize}
\item Oft Teil der Standardinstallation
\item Schlank
\item Device Sync
\end{itemize}
}{banshee}{banshee}{}{}{}
{http://banshee.fm/}

\softwareframe{Clementine}{Einfacher Medienspieler}{
\begin{itemize}
\item Angebunden an Alles
\item Viele Streamingdienste
\item Viele Funktionen
\item Alternativen: \texttt{Amarok, vlc, totem, tomahawk}
\end{itemize}
}{clementine}{clementine}{}{}{}
{https://www.clementine-player.org/de/}