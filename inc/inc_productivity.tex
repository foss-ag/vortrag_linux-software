\section{Produktivität}

%\softwareframe{n}{T}{
%\begin{itemize}
%\item x
%\end{itemize}
%}{p}{p}{}{}{}
%{u}

\softwareframe{Texmaker}{LaTeX Editor}{
\begin{itemize}
\item Gute In\-te\-gra\-tion ver\-schie\-dener Tex Va\-ri\-an\-ten
\item Integrierte PDF-Preview
\end{itemize}
}{texmaker}{texmaker}{}{}{}
{http://www.xm1math.net/texmaker/}

\softwareframe{TeXstudio}{LaTeX Editor}{
\begin{itemize}
\item Gute Autovervollständigung
\end{itemize}
}{texstudio}{texstudio}{}{}{}
{http://www.texstudio.org/}

\softwareframe{Brasero}{Das Gnome Brennprogramm}{
\begin{itemize}
\item Benutzerfreundlich
\item CD/DVD/BR usw.
\item Integrierter Cover-Editor
\item Unterstütze Backends: cdrtools, cdrkit, growisofs und libburn.
\end{itemize}
}{brasero}{brasero}{}{}{}
{https://wiki.gnome.org/Apps/Brasero}

\softwareframe{K3B}{Das KDE Brennprogramm}{
\begin{itemize}
\item KDE Burn Baby, Burn!
\item Umfangreicher als Brasero
\item Viele Funktionen
\item CD/DVD/BR usw.
\item Abbild-Verwaltung
\end{itemize}
}{k3b}{k3b}{}{}{}
{http://k3b.plainblack.com/}

\softwareframe{VirtualBox}{Virtuelle PCs per Knopfdruck}{
\begin{itemize}
\item Nicht-Beschissenes Oracle Produkt 1/2
\item Performante Emulation von x86 32/64bit
\item Software ausprobieren
\item Betriebssysteme ausprobieren
\item Viren ausprobieren
\item Webinterface
\item Snapshots
\item Alternativen: \texttt{qemu + libvirt/virt-manager, VMWare Workstation}
\end{itemize}
}{virtualbox-qt}{virtualbox-qt}{}{}{}
{https://www.virtualbox.org/}

\softwareframe{Marble}{Virtueller Globus}{
\begin{itemize}
\item Nicht hübsch, aber vielfältig
\item Zugriff auf OSM
\item Viele weitere Datenquellen
\item Routing
\end{itemize}
}{marble}{marble}{}{}{}
{https://marble.kde.org/}

\softwaremanframe{Google Earth}{Google's Erde}{
\begin{itemize}
\item Hübsch, aber verkrüppelt
\item Wenig von Google gewartet
\end{itemize}
}{google-earth}{Manueller Download und Installation des .dep-Paketes notwendig.}
{https://www.google.de/intl/de/earth/}

\softwareframe{Transmission}{Eleganter Bittorent-Client}{
\begin{itemize}
\item Client-Daemon Architektur
\item GTK/QT/Mac/Win GUI
\item Webinterface, Konsole
\item Ressourcen-schonend
\end{itemize}
}{transmission-qt}{transmission-qt}{}{}{}
{https://transmissionbt.com/}

\softwaremanframe{yEd}{yEd Grafen und Diagramm Editor}{
\begin{itemize}
\item Grafen-Semantik
\item Clipart direkt suchen
\item Netzwerk-Strukturen
\end{itemize}
}{yed}{Manuelle Installation. Java-Programm, JAR oder fertiges Paket verwenden.}
{https://www.yworks.com/products/yed}

\softwareframe{KRDC}{KDE Remote Desktop Verwaltung}{
\begin{itemize}
\item VNC
\item KDE
\item Nutzt xfreerdp im Hintergrund
\end{itemize}
}{krdc}{krdc}{}{}{}
{https://www.kde.org/applications/internet/krdc/}
