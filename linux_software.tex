% Compile using LuaLaTeX!

\documentclass{beamer}
%\documentclass[handout]{beamer}

% Beamer

%% Optionen
% %\usetheme{Darmstadt} %same as Frankfurt?
% %\usetheme{Frankfurt} % GTI style, bah
% %\usetheme{PaloAlto} %hässlich
% \usetheme{Warsaw} % abgerundet hübsch
% %\usetheme{Luebeck} % wie warsaw, nur eckiger
% 
% \definecolor{beamer@blendedblue}{rgb}{0.5,0.5,0.3} % changed this
% 
% \setbeamercolor{normal text}{fg=black,bg=white}
% \setbeamercolor{alerted text}{fg=red}
% \setbeamercolor{example text}{fg=green!50!black}
% 
% \setbeamercolor{structure}{fg=beamer@blendedblue}
% 
% \setbeamercolor{background canvas}{parent=normal text}
% \setbeamercolor{background}{parent=background canvas}
% 
% \setbeamercolor{palette primary}{fg=yellow,bg=yellow} % changed this
% \setbeamercolor{palette secondary}{use=structure,fg=structure.fg!100!green} % changed this
% \setbeamercolor{palette tertiary}{use=structure,fg=structure.fg!100!green} % changed this

\usetheme{Luebeck} % or split
\usecolortheme{grass}
\setbeamercovered{transparent}

% Bild beim Footer
%\pgfdeclareimage[height=0.4cm]{university-logo}{images/tud_logo_cmyk.pdf}
%\logo{\pgfuseimage{university-logo}}


% more flexible than babel. Requires LuaLaTeX or XeTeX
\usepackage{polyglossia}
\setdefaultlanguage{german}

% PDFLatex? -> fontenc and inputenc
% T1 for western-europe coding
%\usepackage[utf8]{inputenc}
%\usepackage[T1]{fontenc}
%\usepackage{lmodern}

% LuaTeX and XeTeX (but LuaLaTeX is cooler) -> fontspec
\usepackage{fontspec}

% extra symbols
\usepackage{amsmath}
\usepackage{amsfonts}
\usepackage{amssymb}
\usepackage{amsthm}


\usepackage{multimedia}

\usepackage{hyperref}
% Hyperref Optionen - makeatletter/AtBeginDocument erlaubt uns Titel/Author dynamisch zu setzen, wenn gewollt
\makeatletter
	\hypersetup{
	unicode = true,
	pdftoolbar = true,
	pdfmenubar = true,
	pdffitwindow = true,
	pdftitle = {\@title},
	pdfauthor = {\@author},
	pdfsubject = {FOSS-AG},
	colorlinks = true,
	linkcolor = black,
	citecolor = black,
	filecolor = black,
	urlcolor = black
	}
\makeatother



% TOC bei jeder Subsection
% \AtBeginSubsection[]
% {
%   \begin{frame}<beamer>{Outline}
%     \tableofcontents[currentsection,currentsubsection]
%   \end{frame}
% }

% Fußzeile mit Framenumber
\setbeamertemplate{footline}{
	\begin{beamercolorbox}[wd=0.5\textwidth,ht=3ex,dp=1.5ex,leftskip=.5em,rightskip=.5em]{author in head/foot}
		\usebeamerfont{author in head/foot}
		\insertframenumber/\inserttotalframenumber\hfill%\insertshortauthor
	\end{beamercolorbox}
	\vspace*{-4.5ex}\hspace*{0.5\textwidth}
	\begin{beamercolorbox}[wd=0.5\textwidth,ht=3ex,dp=1.5ex,left,leftskip=.5em]{title in head/foot}
		\usebeamerfont{title in head/foot}
		\insertshorttitle
	\end{beamercolorbox}
}


\usepackage[]{graphicx}
\usepackage{listings}
%\lstset{tabsize=2}
% or
%\lstset{
%	numbers = left,
%	frame = tb,
%	mathescape = true,
%	language = Python,
%	escapeinside = {(*@}{@*)},
%	emph = {None, True, False, lock, unlock},
%	emphstyle = {\bfseries},
%	basicstyle = \ttfamily
%}

% Bibliographie
%\usepackage[numbers]{natbib}
%\usepackage{booktabs}
%\usepackage{float}
%\restylefloat{figure}

\usepackage{pdfpages}
\usepackage{color}
\definecolor{tuGreen}{rgb}{0.517,0.721,0.094}
\definecolor{tuOrange}{rgb}{1.0,0.7176,0.0}
\definecolor{brightGray}{gray}{0.9}
\definecolor{darkGray}{gray}{0.2}
\definecolor{white}{rgb}{1,1,1}
\definecolor{black}{rgb}{0,0,0}
\definecolor{red}{rgb}{1,0,0}


% Abkuerzungen richtig formatieren
\usepackage{xspace}
\newcommand{\vgl}{vgl.\xspace} 
\newcommand{\zB}{z.\,B.\xspace}
\newcommand{\dahe}{d.\,h.\nolinebreak[4]\xspace}
\newcommand{\uvm}{u.\,v.\,m.\xspace}
\newcommand{\usw}{u.\,s.\,w.\xspace}
\newcommand{\so}{s.\,o.\xspace}
\newcommand{\sg}{s.\,g.\xspace}
\newcommand{\iA}{i.\,A.\xspace}
\newcommand{\sa}{s.\,a.\xspace}
\newcommand{\su}{s.\,u.\xspace}
\newcommand{\ua}{u.\,a.\xspace}
\newcommand{\og}{o.\,g.\xspace}
\newcommand{\oae}{o.\,\"a.\xspace}
\newcommand{\oBdA}{o.\,B.\,d.\,A.\xspace}
\newcommand{\OBdA}{O.\,B.\,d.\,A.\xspace}

\author{
  Gajda, Michael
  \texttt{michael.gajda@tu-dortmund.de}
  \and
  Test
  Test
}
\date{\today}

\subject{FOSS-AG}
\title{Linux Software für jeden Zweck}
\subtitle{\textcolor{black}{F}ree and \textcolor{black}{O}pen \textcolor{black}{S}ource \textcolor{black}{S}oftware \textcolor{black}{- AG}}
%\institute[TU-Dortmund]{Fakultät für Informatik, Technische Universität Dortmund}
%%\institute[TU-Dortmund]{Department of computer science, Technische Universität Dortmund}


\begin{document}

\begin{frame}
    \titlepage
\end{frame}

\section{Mit Winows zu Linux}

\softwaremanframe{WinSCP}{Dateiaustausch zwischen Windows und Linux}{
\begin{itemize}
\item Wer Angst vor der Kommandozeile hat…
\item Unterstützt SFTP und SCP
\end{itemize}
}{winscp}{Praktisch um Dateien von Zuhause mit der Universität auszutauschen.}
{https://winscp.net/}

\softwaremanframe{Putty}{Windows SSH}{
\begin{itemize}
\item Wer Angst vor GUIs hat…
\item De facto Standard
\item Netzwerktunnel
\end{itemize}
}{putty}{Bitte darauf achten den korrekte Installer herunterzuladen. Angriffs-Gefahr!}
{http://www.putty.org/}

\section{Produktivität}

%\softwareframe{n}{T}{
%\begin{itemize}
%\item x
%\end{itemize}
%}{p}{p}{}{}{}
%{u}

\softwareframe{Texmaker}{LaTeX Editor}{
\begin{itemize}
\item Gute In\-te\-gra\-tion ver\-schie\-dener Tex Va\-ri\-an\-ten
\item Integrierte PDF-Preview
\item Integrierte PDF-Preview
\item Integrierte PDFssssffff
\item Integriertbbbbb
\item Integrssss
\end{itemize}
}{texmaker}{texmaker}{}{}{}
{x}

%\softwareframe{TeXstudio}{LaTeX Editor}{
%\begin{itemize}
%\item x
%\end{itemize}
%}{texstudio}{texstudio}{}{}{}
%{x}

\softwareframe{Brasero}{Das Gnome Brennprogramm}{
\begin{itemize}
\item x
\end{itemize}
}{brasero}{brasero}{}{}{}
{https://wiki.gnome.org/Apps/Brasero}

\softwareframe{K3B}{Das KDE Brennprogramm}{
\begin{itemize}
\item x
\end{itemize}
}{k3b}{k3b}{}{}{}
{http://k3b.plainblack.com/}

\softwareframe{VirtualBox}{Virtuelle PCs per Knopfdruck}{
\begin{itemize}
\item x
\end{itemize}
}{virtualbox-qt}{virtualbox-qt}{}{}{}
{u}

\softwareframe{Marble}{Virtueller Globus}{
\begin{itemize}
\item x
\end{itemize}
}{marble}{marble}{}{}{}
{u}

\softwaremanframe{Google Earth}{Google Earth}{
\begin{itemize}
\item x
\end{itemize}
}{google-earth}{Manueller Download und Installation des .dep-Paketes notwendig}
{https://www.google.de/intl/de/earth/}

\softwareframe{Transmission}{Eleganter Bittorent-Client}{
\begin{itemize}
\item x
\end{itemize}
}{transmission-qt}{transmission-qt}{}{}{}
{https://transmissionbt.com/}

\softwaremanframe{yEd}{yEd Grafen und Diagramm Editor}{
\begin{itemize}
\item x
\end{itemize}
}{yed}{Manuelle Installation. Java-Programm, JAR oder fertiges Paket verwenden}
{https://www.yworks.com/products/yed}

\softwareframe{KRDC}{KDE Remote Desktop Verwaltung}{
\begin{itemize}
\item x
\end{itemize}
}{krdc}{krdc}{}{}{}
{https://www.kde.org/applications/internet/krdc/}


\section{Entwicklung}

\softwareframe{Eclipse}{Die große Java IDE}{
\begin{itemize}
\item Mächtig
\item Hoher Ressourcenverbrauch
\item Pluginsystem vs. Marketplace
\item Viele Plugins
\item Auch manueller Installer 'sauber' $ \rightarrow $ aktuellere Version
\end{itemize}
}{eclipse}{eclipse}{}{}{}
{https://eclipse.org/}

\softwareframe{NetBeans}{Die elegante Java IDE}{
\begin{itemize}
\item Nicht-Beschissenes Oracle Produkt 2/2
\item Übersichtlicher als Eclipse
\item Ähnlich mächtig
\item Vernünftiger GUI-Editor
\item Auch manueller Installer 'sauber' $ \rightarrow $ aktuellere Version
\end{itemize}
}{netbeans}{netbeans}{}{}{}
{https://netbeans.org/}

\softwareframe{QT-Creator}{Entwicklung mit QT}{
\begin{itemize}
\item De facto Standard für QT
\item Schlank
\item Neuste QT-Funktionen
\item C++ Refactoring
\item cmake
\item vagrant
\item Zusätzlich empfohlene Pakete: \texttt{build-essential qttools5-dev-tools qt5-default}
\end{itemize}
}{qtcreator}{qtcreator}{}{}{}
{https://www.qt.io/}

\softwareframe{KDevelop}{KDE Development Tool}{
\begin{itemize}
\item Gute KDE-Integration
\item Schöner Debugger
\end{itemize}
Zusätzlich empfohlene Pakete: \texttt{build-essential}
}{kdevelop}{kdevelop}{}{}{}
{https://www.kdevelop.org/}

\section{Multimeida}
\softwareframe{Blender}{Professionelle 3D-Animation}{
\begin{itemize}
\item x
\end{itemize}
}{blender}{blender}{}{}{}
{https://www.blender.org/}

\softwareframe{Gimp}{Viele Funktionen, gut versteckt}{
\begin{itemize}
\item x
\end{itemize}
}{gimp}{gimp}{}{}{}
{https://www.gimp.org/}

\softwareframe{Inkscape}{Vektor-Grafikprogramm}{
\begin{itemize}
\item x sa OpenToonz
\end{itemize}
}{inkscape}{inkscape}{}{}{}
{https://inkscape.org/}

\softwareframe{OpenSCAD}{3D für Coder}{
\begin{itemize}
\item 3D Drucl
\end{itemize}
}{openscad}{openscad}{}{}{}
{http://www.openscad.org/}

\softwareframe{Audacity}{Soundrecorder}{
\begin{itemize}
\item x
\end{itemize}
}{audacity}{audacity}{}{}{}
{u}

\softwareframe{Ardour}{Mächtige DAW}{
\begin{itemize}
\item x
\end{itemize}
}{ardour}{ardour}{}{}{}
{https://ardour.org/}

\softwareframe{Kodi}{Mediacenter}{
\begin{itemize}
\item x
\end{itemize}
}{kodi}{kodi}{}{}{}
{https://kodi.tv/}

\softwareframe{Darktable}{Ph}{
\begin{itemize}
\item x
\end{itemize}
}{darktable}{darktable}{}{}{}
{http://www.darktable.org/}

\softwareframe{HandBrake}{Video}{
\begin{itemize}
\item x
\end{itemize}
}{handbrake}{handbrake}{}{}{}
{https://handbrake.fr/}

\softwareframe{Kdenlive}{Video}{
\begin{itemize}
\item x
\end{itemize}
}{kdenlive}{kdenlive}{}{}{}
{https://kdenlive.org}

\softwareframe{Gwenview}{Bildbetrachter}{
\begin{itemize}
\item x
\end{itemize}
}{gwenview}{gwenview}{}{}{}
{https://userbase.kde.org/Gwenview}

\softwareframe{Banshee}{Simpler Default Player}{
\begin{itemize}
\item x
\end{itemize}
}{banshee}{banshee}{}{}{}
{u}

\softwareframe{Clementine}{Angebunden an alles}{
\begin{itemize}
\item x sa vlc, totem, tomahawk
\end{itemize}
}{clementine}{clementine}{}{}{}
{u}

\section{Kommunikation}
\softwareframe{Pidgin}{Allround Messenger}{
\begin{itemize}
\item x
\end{itemize}
}{pidgin}{pidgin}{}{}{}
{http://pidgin.im/}

\softwareframe{Mumble}{asd}{
\begin{itemize}
\item x
\end{itemize}
}{mumble}{mumble}{}{}{}
{u}

\softwareframe{xChat}{T}{
\begin{itemize}
\item Sehr einfach
sa Quassel HexChat irssi
\end{itemize}
}{xchat-gnome}{xchat-gnome}{}{}{}
{https://live.gnome.org/Xchat-Gnome/}

\section{Spiele}
\softwareframe{Emulatoren}{RETRO FUN}{
\begin{itemize}
\item Retrogaming ❤ Open Source
\item Spezielles Distros: RetroPi, Emustation
\item Emulatoren:
\item C64: 'frodo'
\item Wii: 'Dolphin'
\item DOS-Spiele: 'DosBox'
\item LukasArt and more: 'ScummVM'
\item SNES: 'zsnes'
\end{itemize}
}{zsnes}{zsnes}{}{}{}
{http://www.zsnes.com/}

\softwareframe{PlayOnLinux}{Wine Manager}{
\begin{itemize}
\item Einfachrer Umgang mit verschiedenen Wine-Versionen
\item Vorkonfigurierte Installer
\item Windows Programme unter Linux
\end{itemize}
}{playonlinux}{playonlinux}{}{}{}
{https://www.playonlinux.com/}

\softwareframe{Steam}{Steam!}{
\begin{itemize}
\item Spiele
\item Spiele
\item Spiele
\item Dez 2016: 2100 Spiele
\end{itemize}
}{steam}{steam}{}{}{}
{http://store.steampowered.com/ https://steamdb.info/linux/}

\section{}

\subsection{}
\begin{frame}
	foo
\end{frame}

\end{document}
