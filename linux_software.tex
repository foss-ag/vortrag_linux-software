% Compile using LuaLaTeX!

\documentclass{beamer}
%\documentclass[handout]{beamer}

% Beamer

%% Optionen
% %\usetheme{Darmstadt} %same as Frankfurt?
% %\usetheme{Frankfurt} % GTI style, bah
% %\usetheme{PaloAlto} %hässlich
% \usetheme{Warsaw} % abgerundet hübsch
% %\usetheme{Luebeck} % wie warsaw, nur eckiger
% 
% \definecolor{beamer@blendedblue}{rgb}{0.5,0.5,0.3} % changed this
% 
% \setbeamercolor{normal text}{fg=black,bg=white}
% \setbeamercolor{alerted text}{fg=red}
% \setbeamercolor{example text}{fg=green!50!black}
% 
% \setbeamercolor{structure}{fg=beamer@blendedblue}
% 
% \setbeamercolor{background canvas}{parent=normal text}
% \setbeamercolor{background}{parent=background canvas}
% 
% \setbeamercolor{palette primary}{fg=yellow,bg=yellow} % changed this
% \setbeamercolor{palette secondary}{use=structure,fg=structure.fg!100!green} % changed this
% \setbeamercolor{palette tertiary}{use=structure,fg=structure.fg!100!green} % changed this
\setbeamerfont{section in sidebar}{size=\fontsize{4}{3}\selectfont}
\setbeamerfont{subsection in sidebar}{size=\fontsize{6}{5.2}\selectfont}

\usetheme{Berkeley} % or split
\usecolortheme{grass}
\setbeamercovered{transparent}

% titelbar kleiner
\makeatletter
\beamer@headheight=1.2\baselineskip
\makeatother

% Bild beim Footer
%\pgfdeclareimage[height=0.4cm]{university-logo}{images/tud_logo_cmyk.pdf}
%\logo{\pgfuseimage{university-logo}}


% more flexible than babel. Requires LuaLaTeX or XeTeX
\usepackage{polyglossia}
\setdefaultlanguage{german}

% PDFLatex? -> fontenc and inputenc
% T1 for western-europe coding
%\usepackage[utf8]{inputenc}
%\usepackage[T1]{fontenc}
%\usepackage{lmodern}

% LuaTeX and XeTeX (but LuaLaTeX is cooler) -> fontspec
\usepackage{fontspec}

% extra symbols
\usepackage{amsmath}
\usepackage{amsfonts}
\usepackage{amssymb}
\usepackage{amsthm}

\usepackage{graphicx}

%% generating qr-codes on the fly! Needs a crazy build command
%\usepackage{pst-barcode, pstricks-add}
%\usepackage{auto-pst-pdf}
%% to make it work with pdflatex and lualatex. Needs an extra '-shell-escape' parameter when compiling

% for strike-through text
\usepackage{soul}

\usepackage{multimedia}

\usepackage{hyperref}
% Hyperref Optionen - makeatletter/AtBeginDocument erlaubt uns Titel/Author dynamisch zu setzen, wenn gewollt
\makeatletter
	\hypersetup{
	unicode = true,
	pdftoolbar = true,
	pdfmenubar = true,
	pdffitwindow = true,
	pdftitle = {\@title},
	pdfauthor = {\@author},
	pdfsubject = {FOSS-AG},
	colorlinks = true,
	linkcolor = black,
	citecolor = black,
	filecolor = black,
	urlcolor = black
	}
\makeatother



% TOC bei jeder Subsection
% \AtBeginSubsection[]
% {
%   \begin{frame}<beamer>{Outline}
%     \tableofcontents[currentsection,currentsubsection]
%   \end{frame}
% }

%% Fußzeile mit Framenumber
%\setbeamertemplate{footline}{
%	\begin{beamercolorbox}[wd=0.5\textwidth,ht=3ex,dp=1.5ex,leftskip=.5em,rightskip=.5em]{author in head/foot}
%		\usebeamerfont{author in head/foot}
%		\insertframenumber/\inserttotalframenumber\hfill%\insertshortauthor
%	\end{beamercolorbox}
%	\vspace*{-4.5ex}\hspace*{0.5\textwidth}
%	\begin{beamercolorbox}[wd=0.5\textwidth,ht=3ex,dp=1.5ex,left,leftskip=.5em]{title in head/foot}
%		\usebeamerfont{title in head/foot}
%		\insertshorttitle
%	\end{beamercolorbox}
%}


\usepackage{listings}
%\lstset{tabsize=2}
% or
%\lstset{
%	numbers = left,
%	frame = tb,
%	mathescape = true,
%	language = Python,
%	escapeinside = {(*@}{@*)},
%	emph = {None, True, False, lock, unlock},
%	emphstyle = {\bfseries},
%	basicstyle = \ttfamily
%}

% Bibliographie
%\usepackage[numbers]{natbib}
%\usepackage{booktabs}
%\usepackage{float}
%\restylefloat{figure}

\usepackage{pdfpages}
\usepackage{color}
\definecolor{tuGreen}{rgb}{0.517,0.721,0.094}
\definecolor{tuOrange}{rgb}{1.0,0.7176,0.0}
\definecolor{brightGray}{gray}{0.9}
\definecolor{darkGray}{gray}{0.2}
\definecolor{white}{rgb}{1,1,1}
\definecolor{black}{rgb}{0,0,0}
\definecolor{red}{rgb}{1,0,0}


% Abkuerzungen richtig formatieren
\usepackage{xspace}
\newcommand{\vgl}{vgl.\xspace} 
\newcommand{\zB}{z.\,B.\xspace}
\newcommand{\dahe}{d.\,h.\nolinebreak[4]\xspace}
\newcommand{\uvm}{u.\,v.\,m.\xspace}
\newcommand{\usw}{u.\,s.\,w.\xspace}
\newcommand{\sg}{s.\,g.\xspace}
\newcommand{\iA}{i.\,A.\xspace}
\newcommand{\sa}{s.\,a.\xspace}
\newcommand{\su}{s.\,u.\xspace}
\newcommand{\ua}{u.\,a.\xspace}
\newcommand{\og}{o.\,g.\xspace}
\newcommand{\oae}{o.\,\"a.\xspace}
\newcommand{\oBdA}{o.\,B.\,d.\,A.\xspace}
\newcommand{\OBdA}{O.\,B.\,d.\,A.\xspace}

\author{
  Gajda, Michael
  \texttt{(michael.gajda@tu-dortmund.de)}
}
\date{\today}

\subject{FOSS-AG}
\title{Linux Software für jeden Zweck}
\subtitle{\textcolor{black}{F}ree and \textcolor{black}{O}pen \textcolor{black}{S}ource \textcolor{black}{S}oftware \textcolor{black}{- AG}}
%\institute[TU-Dortmund]{Fakultät für Informatik, Technische Universität Dortmund}
%%\institute[TU-Dortmund]{Department of computer science, Technische Universität Dortmund}

% defines new command for software frame template
\newcommand{\softwareframe}[9]{
% 1 softwarename
% 2 titletext
% 3 longer desc
% 4 packagename Ubuntu
% 5 packagename Debian
% 6 packagename openSUSE
% 7 packagename Arch
% 8 packagename Gentoo
% 9 url


\subsection{#1}
\begin{frame}{#1}
	\begin{center}
	#2
	\end{center}
	
	\hspace{-0.4cm}
	\begin{minipage}{0.42\linewidth}
	\footnotesize{#3}
	\end{minipage}%
	\hspace{0.02cm}
	\begin{minipage}{0.1\linewidth}
	\includegraphics[scale=0.3]{screenshots/#4}	
	\end{minipage}
	
	\vspace{0.4cm}
	{\scriptsize
	\begin{tabular}{ r l }
	  Ubuntu, Mint und Derivate & \texttt{> apt install #4} \\
	  Debian & \texttt{> apt install #5} \\
	\end{tabular}
	\vspace{0.2cm}
	
	\texttt{URL: #9}
	}
	\vspace{1cm}
\end{frame}
}

\newcommand{\softwaremanframe}[6]{
% 1 softwarename
% 2 titletext
% 3 longer desc
% 4 screenshot
% 5 installnotes
% 6 url


\subsection{#1}
\begin{frame}{#1}
	\begin{center}
	#2
	\end{center}
	
	\hspace{-0.4cm}
	\begin{minipage}{0.42\linewidth}
	\footnotesize{#3}
	\end{minipage}%
	\hspace{0.02cm}
	\begin{minipage}{0.1\linewidth}
	\includegraphics[scale=0.3]{screenshots/#4}	
	\end{minipage}
	
	\vspace{0.4cm}
	{\scriptsize
	#5
	\vspace{0.2cm}
	
	\texttt{URL: #6}
	}
	\vspace{1cm}
\end{frame}
}

\begin{document}

% Titlepage - Hack um Sidebar ohne Inhalt anzuzeigen
\setbeamertemplate{sidebar left}{}
\begin{frame}
    \titlepage
\end{frame}

\begin{frame}{There's \st{an App} a Package for that!}

\begin{center}
{\fontsize{100}{90}\selectfont 68829!}
\end{center}
\vfill

\textcolor{gray}{\tiny (2016-12-01) \\ wget http://packages.ubuntu.com/xenial/allpackages?format=txt.gz -q -O - | zcat | tail -n +6 | wc -l}
\end{frame}


\begin{frame}{Hinweis}
\begin{itemize}
\item Die hier vorgestellte Software ist eine persönliche Auswahl des Vortragenden
\item Vorschläge oder Fragen gewünscht!
\end{itemize}
\end{frame}

\begin{frame}{GPL, Fuck Yeah!}
Fast jede hier vorgestellte Software ist:
\begin{itemize}
\item kostenlos!
\item quelloffen!
\item aus Spaß am Entwickeln entstanden!
\end{itemize}
\end{frame}

% Sidebarinhalt wieder mit TOC aber ohne author
\setbeamertemplate{sidebar left}[sidebar theme]
% Author im Sidebar verstecken
\makeatletter
%  \setbeamertemplate{sidebar \beamer@sidebarside}%{sidebar theme}
%  {
%    \beamer@tempdim=\beamer@sidebarwidth%
%    \advance\beamer@tempdim by -6pt%
%    \insertverticalnavigation{\beamer@sidebarwidth}%
%    \vfill
%    \ifx\beamer@sidebarside\beamer@lefttext%
%    \else%
%      \usebeamercolor{normal text}%
%      \llap{\usebeamertemplate***{navigation symbols}\hskip0.1cm}%
%      \vskip2pt%
%    \fi%
%}%
  \setbeamertemplate{sidebar \beamer@sidebarside}%{sidebar theme}
  {
    \beamer@tempdim=\beamer@sidebarwidth%
    \advance\beamer@tempdim by -6pt%
    \insertverticalnavigation{\beamer@sidebarwidth}%
    \vfill
    \ifx\beamer@sidebarside\beamer@lefttext%
    \else%
      \usebeamercolor{normal text}%
      \llap{\usebeamertemplate***{navigation symbols}\hskip0.1cm}%
      \vskip2pt%
    \fi%
}%
\makeatother

\begin{frame}
Interaktiver Vortrag
\includegraphics[scale=0.5]{images/pollqr}
http://etc.ch/aJrp
\end{frame}

\section{Mit Winows zu Linux}

\subsection{WinSCP}
\begin{frame}
	foo
\end{frame}

\subsection{Putty}
\begin{frame}
	foo
\end{frame}

\section{Produktivität}

%\softwareframe{n}{T}{
%\begin{itemize}
%\item x
%\end{itemize}
%}{p}{p}{}{}{}
%{u}

\softwareframe{TeXstudio}{LaTeX Editor}{
\begin{itemize}
\item Gute Autovervollständigung
\item Vorschau für Bilder und Dateien
\end{itemize}
}{texstudio}{texstudio}{}{}{}
{http://www.texstudio.org/}

\softwareframe{Brasero}{Das Gnome Brennprogramm}{
\begin{itemize}
\item Benutzerfreundlich
\item CD/DVD/BR usw.
\item Integrierter Cover-Editor
\item Unterstütze Backends: cdrtools, cdrkit, growisofs und libburn.
\end{itemize}
}{brasero}{brasero}{}{}{}
{https://wiki.gnome.org/Apps/Brasero}

\softwareframe{K3B}{Das KDE Brennprogramm}{
\begin{itemize}
\item KDE Burn Baby, Burn!
\item Umfangreicher als Brasero
\item Viele Funktionen
\item CD/DVD/BR usw.
\item Abbild-Verwaltung
\end{itemize}
}{k3b}{k3b}{}{}{}
{http://k3b.plainblack.com/}

\softwareframe{VirtualBox}{Virtuelle PCs per Knopfdruck}{
\begin{itemize}
\item Performante Emulation von kompletten PCs
\item Software ausprobieren
\item Betriebssysteme ausprobieren
\item Viren ausprobieren
\item Snapshots
\item Alternativen: \texttt{qemu + libvirt/virt-manager, VMWare Workstation}
\end{itemize}
}{virtualbox-qt}{virtualbox-qt}{}{}{}
{https://www.virtualbox.org/}

\softwareframe{Marble}{Virtueller Globus}{
\begin{itemize}
\item Nicht hübsch, aber vielfältig
\item Zugriff auf OSM
\item Viele weitere Datenquellen
\item Routing
\end{itemize}
}{marble}{marble}{}{}{}
{https://marble.kde.org/}

\softwaremanframe{Google Earth}{Google's Erde}{
\begin{itemize}
\item Hübsch, aber verkrüppelt
\item Wenig von Google gewartet
\end{itemize}
}{google-earth}{Manueller Download und Installation des .dep-Paketes notwendig.}
{https://www.google.de/intl/de/earth/}

\softwareframe{Transmission}{Eleganter Bittorent-Client}{
\begin{itemize}
\item Client-Daemon Architektur
\item Webinterface, Konsole
\item Ressourcen-schonend
\end{itemize}
}{transmission-qt}{transmission-qt}{}{}{}
{https://transmissionbt.com/}

\softwaremanframe{yEd}{yEd Grafen und Diagramm Editor}{
\begin{itemize}
\item Grafen-Semantik
\item Clipart direkt suchen
\item Netzwerk-Strukturen
\end{itemize}
}{yed}{Manuelle Installation. Java-Programm, JAR oder fertiges Paket verwenden.}
{https://www.yworks.com/products/yed}

\softwareframe{KRDC}{KDE Remote Desktop Verwaltung}{
\begin{itemize}
\item VNC
\item KDE
\end{itemize}
}{krdc}{krdc}{}{}{}
{https://www.kde.org/applications/internet/krdc/}


\section{Entwicklung}
\softwareframe{NetBeans}{Java IDE}{
\begin{itemize}
\item x
\end{itemize}
}{netbeans}{netbeans}{}{}{}
{https://netbeans.org/}

\softwareframe{n}{T}{
\begin{itemize}
\item x
\end{itemize}
}{p}{p}{}{}{}
{u}

\softwareframe{n}{T}{
\begin{itemize}
\item x
\end{itemize}
}{p}{p}{}{}{}
{u}

\section{Multimeida}
\softwareframe{Blender}{Professionelle 3D-Animation}{
\begin{itemize}
\item x
\end{itemize}
}{blender}{blender}{}{}{}
{https://www.blender.org/}

\softwareframe{Gimp}{Viele Funktionen, gut versteckt}{
\begin{itemize}
\item x
\end{itemize}
}{gimp}{gimp}{}{}{}
{https://www.gimp.org/}

\softwareframe{Inkscape}{Vektor-Grafikprogramm}{
\begin{itemize}
\item x sa OpenToonz
\end{itemize}
}{inkscape}{inkscape}{}{}{}
{https://inkscape.org/}

\softwareframe{OpenSCAD}{3D für Coder}{
\begin{itemize}
\item 3D Drucl
\end{itemize}
}{openscad}{openscad}{}{}{}
{http://www.openscad.org/}

\softwareframe{Audacity}{Soundrecorder}{
\begin{itemize}
\item x
\end{itemize}
}{audacity}{audacity}{}{}{}
{u}

\softwareframe{Ardour}{Mächtige DAW}{
\begin{itemize}
\item x
\end{itemize}
}{ardour}{ardour}{}{}{}
{https://ardour.org/}

\softwareframe{Kodi}{Mediacenter}{
\begin{itemize}
\item x
\end{itemize}
}{kodi}{kodi}{}{}{}
{https://kodi.tv/}

\softwareframe{Darktable}{Ph}{
\begin{itemize}
\item x
\end{itemize}
}{darktable}{darktable}{}{}{}
{http://www.darktable.org/}

\softwareframe{HandBrake}{Video}{
\begin{itemize}
\item x
\end{itemize}
}{handbrake}{handbrake}{}{}{}
{https://handbrake.fr/}

\softwareframe{Kdenlive}{Video}{
\begin{itemize}
\item x
\end{itemize}
}{kdenlive}{kdenlive}{}{}{}
{https://kdenlive.org}

\softwareframe{Gwenview}{Bildbetrachter}{
\begin{itemize}
\item x
\end{itemize}
}{gwenview}{gwenview}{}{}{}
{https://userbase.kde.org/Gwenview}

\softwareframe{Banshee}{Simpler Default Player}{
\begin{itemize}
\item x
\end{itemize}
}{banshee}{banshee}{}{}{}
{u}

\softwareframe{Clementine}{Angebunden an alles}{
\begin{itemize}
\item x sa vlc, totem, tomahawk
\end{itemize}
}{clementine}{clementine}{}{}{}
{u}

\section{Kommunikation}
\softwareframe{Pidgin}{Allround Messenger}{
\begin{itemize}
\item De factor Standard
\item Viele Protokolle
\item Plugin-Support für weitere Protokolle
\end{itemize}
}{pidgin}{pidgin}{}{}{}
{http://pidgin.im/}

\softwareframe{Mumble}{Effizienter Voice-Chat}{
\begin{itemize}
\item Offene Alternative zu TeamSpeak
\item Speex
\item Positions–Audio
\end{itemize}
}{mumble}{mumble}{}{}{}
{https://wiki.mumble.info/}

\softwareframe{XChat}{IRC Client}{
\begin{itemize}
\item Sehr einfach
\item Alternativen: \texttt{Quassel, HexChat, irssi}
\end{itemize}
}{xchat-gnome}{xchat-gnome}{}{}{}
{https://live.gnome.org/Xchat-Gnome/}

\section{Spiele}
\softwareframe{ZSNES}{SNES FUN}{
\begin{itemize}
\item x
sa frodo, dolphin, DosBox ScummVM
\end{itemize}
}{zsnes}{zsnes}{}{}{}
{http://www.zsnes.com/}

\softwareframe{PlayOnLinux}{Wine Manager}{
\begin{itemize}
\item x
\end{itemize}
}{playonlinux}{playonlinux}{}{}{}
{https://www.playonlinux.com/}

\softwareframe{Steam}{Steam!}{
\begin{itemize}
\item x
\end{itemize}
}{steam}{steam}{}{}{}
{http://store.steampowered.com/}

\include{inc/inc_advanced}

\end{document}
